\documentclass[11pt,a4paper]{article}
\usepackage[hyperref]{acl2020}
\usepackage{times}
\usepackage{latexsym}
\renewcommand{\UrlFont}{\ttfamily\small}
\usepackage{microtype}
\usepackage[autostyle=false, style=english]{csquotes}
\MakeOuterQuote{"}

\aclfinalcopy % Uncomment this line for the final submission
%\def\aclpaperid{***} %  Enter the acl Paper ID here

\newcommand\BibTeX{B\textsc{ib}\TeX}

\title{Laughter Through the Decades: Replicating Humor Detection Research}

\author{Joshua Hoeflich \\
  Northwestern University / 1881 Oak Avenue, Evanston IL \\
  \texttt{joshuahoeflich2021@u.northwestern.edu} \\
  }

\date{}

\begin{document}
\maketitle
\begin{abstract}
Please write an abstract that explains what this final project is about.
\end{abstract}

\section{Introduction}
Please write an introduction that explains what this final project is about.

\section{Creating Comparable Data Sets}
Replicating the humor detection systems described in the paper first required finding appropriate data to use on the models described in the article. Unfortunately, as of 2021, the exact data sets used in "Making Computers Laugh" are unavailable. The paper describes an elaborate web scraping process used to collect 16000 one-liners, but the links it uses to describe that method no longer point to live websites. The paper also does not link to the "online proverb collection" it uses for examples of non-humorous input data; it does not describe precisely which Reuters headlines or British National Corpus sentences it uses.

Accordingly, our first task involved finding data different but comparable to that used in the article.

\section{Replicating Humor Recognition}

\section{Experimental Results}

\section{Conclusion}

\end{document}
