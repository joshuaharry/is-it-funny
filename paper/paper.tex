\documentclass[11pt,a4paper]{article}
\usepackage[hyperref]{acl2020}
\usepackage{times}
\usepackage{latexsym}
\usepackage{tabularx}
\usepackage{makecell}
\renewcommand{\UrlFont}{\ttfamily\small}
\usepackage{microtype}
\usepackage[autostyle=false, style=english]{csquotes}
\MakeOuterQuote{"}

\aclfinalcopy % Uncomment this line for the final submission
%\def\aclpaperid{***} %  Enter the acl Paper ID here

\newcommand\BibTeX{B\textsc{ib}\TeX}

\title{Laughter Through the Decades: Replicating Humor Detection Research}

\author{Joshua Hoeflich \\
  Northwestern University / 1881 Oak Avenue, Evanston IL \\
  \texttt{joshuahoeflich2021@u.northwestern.edu} \\
  }

\date{}

\begin{document}
\maketitle
\begin{abstract}
  We attempt to replicate the results of "Making Computers Laugh: Investigations in Automatic Humor Recognition" by applying the techniques within it for classifying humorous and non-humorous texts to publicly available data sets. \cite{mihalcea2005making} We compare the performance of three models: A naive Bayes classifier, an SVM classifier, and a decision tree classifier that uses heuristics based on the presence of antonyms, adult slang, and sounds such as rhymes and alliteration within documents. We train and compare these models on a collection of short jokes, Jeopardy questions, news headlines, and Amazon food reviews. We find that despite substantive differences in the datasets and implementations in our work and the original paper, our systems perform comparably well at the same tasks.
\end{abstract}

\section{Introduction}
Few other tasks illustrate the astonishing effectiveness of machine learning as humor recognition. Comedy seems so subjective and points to so many unanswered deep philosophical questions that, at first glance, the notion of teaching a machine to recognize jokes appears absurd. However, in 2005, computational linguists Rada Mihalcea and Carlo Strapparava published a paper titled "Making Computers Laugh: Investigations in Automatic Humor Recognition" showing that Naive Bayes and Support Vector Machine (SVM) classifiers could do a remarkably good job at distingushing one-liners from many other kinds of texts. The strength of their results merits deeper examination.

In this article, we describe an attempt to replicate the results of "Making Computers Laugh" over 15 years after its original publication. We describe the steps involved with mimicking their processes and the challenges we faced while trying to do so, taking special note of the places where directly using the methods described in the original work proved infeasible. We then evaluate our new results in comparison with the original ones and conclude with a discussion of relevant related works.

\section{Data: Creating Comparable Sets}
Replicating the humor detection systems described in the paper first required finding appropriate data to use on the models described in the article. Unfortunately, as of 2021, the exact data sets used in "Making Computers Laugh" are unavailable. The paper describes an elaborate web scraping process used to collect 16000 one-liners, but the links it uses to describe that method no longer point to live websites. The paper also does not link to the "online proverb collection" it uses for examples of non-humorous input data; it does not describe precisely which Reuters headlines or British National Corpus sentences it uses.

Accordingly, our first task involved finding data different but comparable to that used in the article. For positive examples of humor, we used a collection of over 200,000 short jokes; for negative examples, we used a list of 200,000 Jeopardy questions, a list of 1,000,000 news headlines from the Australian Broadcasting Corporation (ABC), and 500,000 reviews taken from Amazon.com about fine foods. \cite{short-jokes} \cite{news-headlines} \cite{jeopardy-questions} \cite{amazon-reviews} All of these datasets came from kaggle, a website with numerous other extensive datasets and machine learning models.

Using short jokes instead of the one liners in the original paper is appropriate because both forms of data work well with classifiers for the same reason. The original authors note: "While longer jokes can have a relatively complex narrative structure, a one-liner must produce the humorous effect `in one shot', with very few words. These characteristics make this type of humor particularly suitable for use in an automatic learning setting" (2005). The short jokes selected for this replication are not necessarily guaranteed to be exactly one sentence long, but their similar length means that our system will likely perform well or badly on them for reasons analagous to those explored in the original paper.

\section{Methods: Replicating Humor Recognition}
With several datasets at hand, the next step of replicating the paper required creating and training several different machine learning models. We started by creating a decision tree classifier that relied on embedding documents into vectors with numeric components derived from up to three heuristics: the count of antonyms, the count of words related to sexuality, and the count of alliterations and rhymes found in a given document. No off-the-shelf mapping between those heuristics and vectors existed, so we needed to implement our own based on the content of the paper.

Finding antonyms involved using the semantic information stored in Princeton's WordNet lexical database. The paper used this dataset as well, though they may have relied on an older version; we used the one directly available through version 3.6.5 of the Natural Language Toolkit (NLTK) Python library. \cite{10.5555/1717171} We also used the version of Carnegie Mellon's pronunciation dictionary bundled with NLTK to count the number of rhymes and alliterations within a sentence. Implementing these counts required writing a few functions that tokenized the sentences, stripped out their stopwords, and tracked the relevant features of each heuristic using the APIs that NLTK exposes directly.

Tracking the words related to sexuality proved more challenging. "Making Computers Laugh" relied on traversing a graph constructed with help from the WordNet Domains project, which associated items from WordNet with relevant domain labels. More specifically, the authors performed a search by finding words related to one another through the \textsc{sexuality} label in that project. Unfortunately, the link to download WordNet Domains no longer appears to work; while an unofficial copy of the project appears to exist on GitHub, the data within it appears out of date with the most modern WordNet corpus. Because of these challenges, we opted to find words related to sexuality differently: We relied on pre-computed GloVe vectors for word representation. \cite{pennington-etal-2014-glove} We wrote a small C++ program to find the top 5,000 words with the closest cosine similarity to the word \textsc{sexuality} and counted the number of times they appeared within a document. As we shall see in the results section, this strategy appears to have yielded analagous results to those in the original paper.

After implementing these heuristics, we had the information we needed to begin running the three machine learning models. The scikit-learn library provided both high-quality off-the-shelf tools for transforming raw documents into TFIDF vectors and using those vectors to learn with Naive Bayes and an SVM classifiers. \cite{sklearn_api} Replicating the paper in those circumstances simply involved running those tools.

\section{Results}

Our results using the decision tree classifier with vectors extracted from the heuristics are below. We trained the model using 1000 samples and tested it on 15000 taken from each of the three datasets; the mean accuracy on the test sets are reported below.

\begin{center}
  \begin{tabular}{|p{1.5cm}|p{1.5cm}|p{1.5cm}|p{1.5cm}|}
    \hline
    Heuristic & \thead{Jokes/ \\ Reviews} & \thead{Jokes/ \\ Jeopardy} & \thead{Jokes/ \\ Headlines} \\
    \hline
    Adult Slang & 71.32\% & 60.32\% & 69.06\% \\
    \hline
    Sounds & 65.23\% & 52.67\% & 57.77\% \\
    \hline
    Antonyms & 51.92\% & 51.31\% & 52.07\% \\
    \hline
    All & 73.82\% & 60.16\% & 70.21\% \\
    \hline
\end{tabular}
\end{center}

\noindent \textsc{Table 1}: Humor recognition mean accuracies using adult slang words, sounds (such as rhymes and alliteration), antonyms, and all heuristics at once.\\

This range of percentages (70\% to 50\%) match roughly the range found in the original paper.

We now describe our results using Naive Bayes and SMV learning. In these examples, we also used 1000 training samples and 15000 test samples. The results were as follows:

\begin{center}
  \begin{tabular}{|p{1.5cm}|p{1.5cm}|p{1.5cm}|p{1.5cm}|}
    \hline
    Technique & \thead{Jokes/ \\ Reviews} & \thead{Jokes/ \\ Jeopardy} & \thead{Jokes/ \\ Headlines} \\
    \hline
    Naive Bayes & 90.82\% & 88.81\% & 94.35\% \\
    \hline
    SVM & 93.37\% & 92.99\% & 97.13\% \\
    \hline
\end{tabular}
\end{center}

\noindent \textsc{Table 2}: Humor recognition mean accuracies using Naive Bayes and SVM classifiers.\\

These scores are comparable to the ones in the original paper, albeit somewhat higher in the worst-case scenarios.

\section{Discussion}
Intriguingly, in our experiment, we found that words related to sexuality served as the most effective indicator of humorous content, whereas in the original paper alliteration and rhymes yielded the most accurate clues. Differences in the jokes dataset may have caused this discrepancy; jokes explicitly labeled as one-liners are probably more likely to rely on sound based humor than those that are specifically short. Furthermore, our decision to use GloVe vectors instead of manually labelled documents may have contributed to the effectiveness of that measure as well.

Given the numerous differences in implementation within this project and the original text, that these content-based classifiers achieved equally good or even better results suggests that the techniques and findings used in "Making Computers Laugh" do indeed replicate: classic classification techniques can in some circumstances successfully distinguish between short humorous and non-humorous documents from various corpora.

\section{Related Work}

Other researchers have cited "Making Computers Laugh" in further work in humor recognition. For example, in "That's What She Said: Double Entendre Identification", the authors used the paper as inspiration for a system that could detect whether saying "That's what she said" would make a sentence funny or not (Kiddon and Brun, 2011) \cite{kiddon2011s}. In "Humor: Prosody Analysis and Automatic Recognition for F * R * I * E * N * D * S *", the authors used the paper as inspiration for a system that applied classifiers to the scripts of the T.V. show \emph{Friends} annotated with laugh tracks to detect which lines were and were not funny (Purandare and Litman, 2006) \cite{purandare2006humor}.

Computational linguists working on problems other than humor recognition have found "Making Computers Laugh" helpful as well. For example, in "Semi-Supervised Recognition of Sarcastic Sentences in Twitter and Amazon", the authors took inspiration from the Naive Bayes and SVM classifiers when trying to implement a system that could detect sarcasm in documents (Davidov, Tsur, and Rapport, 2010) \cite{davidov2010semi}. Moreover, in "Automatic Detection of Fake News", the authors cite the paper when describing the challenges outlets like \emph{The Onion} pose when trying to detect nonfactual news stories (P{\'e}rez-Rosas et. al, 2017) \cite{perez2017automatic}. That "Making Computers Laugh" had an impact on articles grappling with the enormous influence of social media illustrates the paper's importance. By helping others understand some of the most contentious and important language of our time, "Making Computers Laugh" shows that humor recognition is no laughing matter.

\bibliographystyle{plain}
\bibliography{acl2020.bib}
\end{document}
